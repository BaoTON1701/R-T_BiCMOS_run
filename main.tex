\documentclass[10pt, a4paper]{article}

% --- Packages ---
% \usepackage[margin=0.75in, top=1in]{geometry} % Adjust margins
% Tight margins for a datasheet look
\usepackage[left=0.75in, right=0.75in, top=0.5in, bottom=0.75in]{geometry}
\usepackage{graphicx}       % For the logo
\usepackage{xcolor}         % For colors
\usepackage{multicol}       % For two-column layout
\usepackage{titlesec}       % For custom section headers
\usepackage{enumitem}       % For list formatting
\usepackage{lmodern}       % Times New Roman-like font for body
\usepackage{helvet}         % Helvetica for headers
\usepackage{circuitikz}     % For the circuit diagram
\usepackage{tikz}           % For general drawing
\usepackage{booktabs}       % For nicer tables
\usepackage{multirow}       % For merging table cells
\usepackage{tabularx}       % For flexible tables
\usepackage{amsmath}        % For math symbols like \pm
\usepackage{caption}        % Required for \captionof
\usepackage{float}          % For [H] placement specifier
\usepackage{todonotes}

% --- Custom Colors ---
% Approximating the "Analog Devices" red/maroon
\definecolor{headerred}{RGB}{160, 20, 30}

% --- Font & Section Setup ---
% Define a custom font command for the big headers
\newcommand{\datasheetheader}[1]{%
    {\fontfamily{phv}\selectfont\bfseries\color{headerred}\LARGE\MakeUppercase{#1}}%
}

% Customize standard \section to look like the datasheet headers
\titleformat{\section}
  {\fontfamily{phv}\selectfont\bfseries\color{headerred}\Large\MakeUppercase} % Format
  {} % Label
  {0pt} % Sep
  {} % Before code

\titlespacing*{\section}{0pt}{10pt}{5pt}
% This defines "C" as a Centered version of the "X" column
\newcolumntype{C}{>{\centering\arraybackslash}X}
% Remove paragraph indentation
\setlength{\parindent}{0pt}

%
% --- Header & Footer Setup ---
\usepackage{fancyhdr}
\pagestyle{fancy} % Turn on the fancy style
\fancyhf{} % Clear all default headers and footers

% 1. Define the thick colored Header Line
\renewcommand{\headrule}{%
    {\color{headerred}\hrule height 2pt}%
    \vspace{2pt} % Add a little space between line and body text
}

% 2. Define the thick colored Footer Line
\renewcommand{\footrule}{%
    \vspace{2pt} % Add a little space between body text and line
    {\color{headerred}\hrule height 2pt}%
}

% 3. Set the text content (Optional)
\fancyhead[L]{\small \textbf{R\&T BiCMOS}}    % Top Left
\fancyhead[R]{\small \textbf{B.TON - APC}} % Top Right
\fancyfoot[C]{\thepage}                   % Bottom Center (Page Number)

% --- Fix for head height warning ---
% Because the line is 7pt thick, you might need to increase headheight
\setlength{\headheight}{25pt}
%


\begin{document}

% ==========================================
% HEADER SECTION
% ==========================================
\begin{minipage}[b]{0.4\textwidth}
    % Placeholder for Logo. Replace example-image with your logo file.
    \includegraphics[width=1.5cm]{image/logo/RetT_black.PNG} \\
    {\fontfamily{phv}\selectfont\bfseries\Huge\hspace{3pt} \scalebox{1.0}{R\&T BiCMOS run2}}
\end{minipage}
\hfill
\begin{minipage}[b]{0.55\textwidth}
    \raggedleft
    % {\fontfamily{phv}\selectfont\Huge\bfseries Run2}\\[5pt]
    {\fontfamily{phv}\selectfont\Huge Low Noise, Cryogenic \\ Differential Amplifier}
\end{minipage}

\vspace{5pt}
{\color{headerred}\hrule height 2pt}
\vspace{10pt}
Prepared by: Bao TON

% ==========================================
% TWO-COLUMN BODY
% ==========================================
\begin{multicols}{2}

    % --- DESCRIPTION ---
    \section*{Description}
This ASIC, fabricated using IHP technology, integrates three distinct circuits designed for cryogenic operation ($77\,\text{K}$). It features two low-noise differential amplifiers: a fully integrated version with on-chip resistors and a flexible version utilizing external resistors. Both amplifiers deliver an ultralow noise floor of $1\,\text{nV}/\sqrt{\text{Hz}}$ and are optimized for promising low flicker noise at cryogenic temperatures. The bandwidth is specified at $25\,\text{MHz}$ for the on-chip variant and $50\,\text{MHz}$ for the external resistor configuration. Additionally, a large-geometry NMOS transistor ($W/L \approx 190,000$) is included for discrete characterization.

    % --- ABSOLUTE MAXIMUM RATINGS ---
    \section*{Absolute Maximum Ratings}
    
    
    \begin{tabbing}
        \hspace{6cm} \= \kill
        Total Supply Voltage ($\text{V}^+$ to $\text{V}^-$) \> 5.15V \\
        \dotfill \\
        Input Current (Note 2) \> $\pm$40mA \\
        \dotfill \\
        Operating Junction Temperature Range \\
        \hspace{0.5cm} (Note 5) \> $-40^\circ$C to $125^\circ$C \\
       
    \end{tabbing}
  

\end{multicols}

\vspace{5pt}
{\color{headerred}\hrule height 2pt}
\vspace{10pt}

\begin{multicols}{2}

    % --- BURN-IN CIRCUIT ---
    \section*{Block diagram}
    \vspace{0.5cm}
    \begin{center}
    \begin{circuitikz}[scale=0.8, transform shape]
        % Paths, nodes and wires:
	\node[fd op amp, yscale=-1] at (2.44, 28.99){};
	\node[nmosd, xscale=-1] at (10.047, 29){};
	\draw (2, 29.75) -- (2, 31.25);
	\draw (2, 28.25) -- (2, 28);
	\node[shape=rectangle, minimum width=1.09cm, minimum height=0.445cm] at (0.687, 29.72){} node[anchor=north west, align=left, text width=0.702cm, inner sep=6pt] at (0.125, 29.96){$V_{in1+}$ };
	\node[shape=rectangle, minimum width=1.09cm, minimum height=0.445cm] at (0.687, 28.74){} node[anchor=north west, align=left, text width=0.702cm, inner sep=6pt] at (0.125, 28.98){$V_{in1-}$ };
	\node[shape=rectangle, minimum width=1.09cm, minimum height=0.445cm] at (2.1, 31.69){} node[anchor=north west, align=left, text width=0.702cm, inner sep=6pt] at (1.537, 31.93){$V_{cc}$ };
	\node[shape=rectangle, minimum width=1.09cm, minimum height=0.445cm] at (2.187, 26.76){} node[anchor=north west, align=left, text width=0.702cm, inner sep=6pt] at (1.625, 27){$V_{ee}$ };
	\draw (2, 28) to[american resistor, /tikz/circuitikz/bipoles/length=0.700cm, l={$R_{ptat1}$}] (2, 27.25);
	\draw (2, 27.25) -| (2, 27);
	\node[shape=rectangle, minimum width=1.09cm, minimum height=0.425cm] at (3.835, 29.71){} node[anchor=north west, align=left, text width=0.702cm, inner sep=6pt] at (3.273, 29.94){$V_{out1-}$ };
	\node[shape=rectangle, minimum width=1.09cm, minimum height=0.445cm] at (3.835, 28.74){} node[anchor=north west, align=left, text width=0.702cm, inner sep=6pt] at (3.273, 28.98){$V_{out1+}$ };
	\draw (6.5, 31.25) -- (7, 31.25);
	\node[shape=rectangle, minimum width=1.09cm, minimum height=0.445cm] at (11.859, 29.01){} node[anchor=north west, align=left, text width=0.702cm, inner sep=6pt] at (11.297, 29.25){$G_{NMOS}$ };
	\node[shape=rectangle, minimum width=1.09cm, minimum height=0.445cm] at (9.859, 30.49){} node[anchor=north west, align=left, text width=0.702cm, inner sep=6pt] at (9.297, 30.73){$D_{NMOS}$ };
	\node[fd op amp, yscale=-1] at (7.19, 28.97){};
	\draw (6.5, 29.862) -- (6.5, 30);
	\draw (6.75, 28.23) -- (6.75, 27.98);
	\node[shape=rectangle, minimum width=1.09cm, minimum height=0.445cm] at (5.437, 29.7){} node[anchor=north west, align=left, text width=0.702cm, inner sep=6pt] at (4.875, 29.94){$V_{in2+}$ };
	\node[shape=rectangle, minimum width=1.09cm, minimum height=0.445cm] at (5.437, 28.72){} node[anchor=north west, align=left, text width=0.702cm, inner sep=6pt] at (4.875, 28.96){$V_{in2-}$ };
	\node[shape=rectangle, minimum width=1.09cm, minimum height=0.445cm] at (6.862, 31.69){} node[anchor=north west, align=left, text width=0.702cm, inner sep=6pt] at (6.3, 31.93){$V_{cc}$ };
	\draw (6.75, 27.98) to[american resistor, /tikz/circuitikz/bipoles/length=0.700cm, l={$R_{ptat2}$}] (6.75, 27.23);
	\node[shape=rectangle, minimum width=1.09cm, minimum height=0.425cm] at (8.585, 29.69){} node[anchor=north west, align=left, text width=0.702cm, inner sep=6pt] at (8.023, 29.92){$V_{out2-}$ };
	\node[shape=rectangle, minimum width=1.09cm, minimum height=0.445cm] at (8.585, 28.72){} node[anchor=north west, align=left, text width=0.702cm, inner sep=6pt] at (8.023, 28.96){$V_{out2+}$ };
	\draw (6.5, 31) -- (7, 31);
	\draw (10.047, 28.23) -| (10.047, 26.98);
	\draw (6.75, 27.23) -- (6.75, 27);
	\draw (1.75, 27) -- (2.25, 27);
	\draw (6.5, 27) -- (7, 27);
	\node[shape=rectangle, minimum width=1.09cm, minimum height=0.445cm] at (6.938, 26.74){} node[anchor=north west, align=left, text width=0.702cm, inner sep=6pt] at (6.375, 26.98){$V_{ee}$ };
	\draw (9.797, 27) -- (10.297, 27);
	\node[shape=rectangle, minimum width=1.09cm, minimum height=0.445cm] at (10.234, 26.75){} node[anchor=north west, align=left, text width=0.702cm, inner sep=6pt] at (9.672, 26.99){$V_{ee}$ };
	\node[shape=rectangle, minimum width=1.465cm, minimum height=0.465cm] at (2, 25.5){} node[anchor=north west, align=left, text width=1.077cm, inner sep=6pt] at (1.25, 25.75){LNA 1};
	\node[shape=rectangle, minimum width=1.465cm, minimum height=0.465cm] at (6.75, 25.5){} node[anchor=north west, align=left, text width=1.077cm, inner sep=6pt] at (6, 25.75){LNA 2};
	\node[shape=rectangle, minimum width=2.465cm, minimum height=0.465cm] at (10.297, 25.5){} node[anchor=north west, align=left, text width=2.077cm, inner sep=6pt] at (9.047, 25.75){NMOS Block};
	\draw (7, 29.575) -- (7, 30);
	\draw (6.5, 30.75) to[american resistor, /tikz/circuitikz/bipoles/length=0.700cm, l_={$R_{L+}$}] (6.5, 30);
	\draw (7, 30.75) to[american resistor, /tikz/circuitikz/bipoles/length=0.700cm, l={$R_{L-}$}] (7, 30);
	\draw (6.5, 30.75) -| (6.5, 31) -- (7, 31) -| (7, 30.75);
	\draw (6.75, 31.25) -| (6.75, 31);
	\draw (1.75, 31.25) -- (2.25, 31.25);
	\node[ocirc] at (10.047, 30){};
	\node[ocirc] at (11.347, 29){};
	\draw (10.047, 29.701) -- (10.047, 29.951);
	\draw (11.027, 29) -- (11.297, 29);
    \end{circuitikz}
    \end{center}

    % --- PACKAGE INFO ---
    \section*{Package}

    
    \begin{center} % Use center instead of figure
        % \linewidth refers to the width of the specific COLUMN, not the page
        \includegraphics[width=0.8\linewidth]{image/packaging/BondingRTrun2.PNG}
        
        % Use \captionof{figure}{...} instead of \caption{...}
        \captionof*{figure}{Bonding diagram with QFN24 package} 
        \label{fig:placeholder}
    \end{center}

\end{multicols}
\vspace{5pt}
{\color{headerred}\hrule height 2pt}
\vspace{10pt}
\section*{TABLE 1: ELECTRICAL CHARACTERISTICS}
\renewcommand{\arraystretch}{1.2}
\begin{table}[ht]
    \centering
    % Resizebox ensures the table fits the page regardless of column widths
    \resizebox{\textwidth}{!}{%
    
        % Note: I changed 'C' to 'm{3.5cm}' in the line below
        \begin{tabularx}{1.55\textwidth}{l|l|m{3.5cm}|c|ccc|c|ccc|c|c}
            \hline
            \multirow{2}{*}{\textbf{SYMBOL}} & 
            \multirow{2}{*}{\parbox{4cm}{\centering \textbf{PARAMETER}}} & 
            \multirow{2}{*}{\centering \textbf{CONDITIONS}} & % Added \centering here manually
            \multirow{2}{*}{\textbf{NOTES}} & 
            \multicolumn{3}{c|}{$\mathbf{T_A = 27^\circ C}$} & 
            \multirow{2}{*}{\parbox{1.5cm}{\centering \textbf{SUB-}\\ \textbf{GROUP}}} & 
            \multicolumn{3}{c|}{$\mathbf{-55^\circ C \le T_A \le 125^\circ C}$} & 
            \multirow{2}{*}{\parbox{1.5cm}{\centering \textbf{SUB-}\\ \textbf{GROUP}}} & 
            \multirow{2}{*}{\textbf{UNITS}} \\
            
            \cline{5-7} \cline{9-11}
             & & & & \textbf{MIN} & \textbf{TYP} & \textbf{MAX} & & \textbf{MIN} & \textbf{TYP} & \textbf{MAX} & & \\
            \hline
            
            $V_{OS}$ & Input Offset & \centering $V_S = \pm 1.65\text{V}$ & & & 0.6 & 2 & 1 & & & 4 & 2,3 & mV \\
                     & Voltage      & \centering $V_{CM} = V^-\text{ to }V^+$ & & & 2.5 & 6 & 1 & & & 9 & 2,3 & mV \\
            \hline
            $I_B$    & Input Bias   & \centering $V_S = \pm 1.65\text{V}$ & & & 8 & 18 & 1 & & & 20 & 2,3 & $\mu$A \\
                     & Current      & \centering $V_{CM} = V^+$               & & -50 & -23 & & 1 & -100 & & & 2,3 & $\mu$A \\
            \hline
                & Input Noise   & \centering $0.1~Hz$ to $10~Hz$ & & & 8 & 18 & 1 & & & 20 & 2,3 & $\mu$A \\
                     & Voltage      & \centering $V_{CM} = V^+$               & & -50 & -23 & & 1 & -100 & & & 2,3 & $nV_{p-p}$ \\
            \hline
            SR    & Slew Rate   & \centering $V_{in} \pm 1.25~m\text{V}$ | LNA1 & & 28 & 30 & 31 & 1 & 29 & 30 & 32 & 2,3 & $V/\mu s$ \\
                     &       & \centering $V_{in} \pm 1.25~m\text{V}$ | LNA2    & & 39 & 36 & 42 & 1 & 38 & 39 & 40 & 2,3 & $/\mu s $ \\
        \end{tabularx}%
    }
\end{table}
\newpage
\subsection*{Slew Rate Measurement}
\begin{multicols}{2}
    \begin{figure}[H]
        \centering
        \includegraphics[width=\linewidth]{image/plot/Output_Response_27degC.png} 
        \captionof{figure}{Output Response}
    \end{figure}

    \begin{figure}[H]
        \centering
        \includegraphics[width=\linewidth]{image/plot/Slew_rate.png}
        \captionof{figure}{Slew Rate}
    \end{figure}

    \begin{figure}[H]
        \centering
        \includegraphics[width=\linewidth]{image/plot/Slew_rate_estimate.png}
        \captionof{figure}{Slew Rate Estimation}
    \end{figure}

    \begin{figure}[H]
        \centering
        \includegraphics[width=\linewidth]{image/plot/Hist_slew_rate.png}
        \captionof{figure}{Slew Rate Histogram}
    \end{figure}
\end{multicols}
\pagebreak
\subsection*{Common-mode Rejection Ratio (CMRR)}
\begin{multicols}{2}
    \begin{figure}[H]
        \centering
        \includegraphics[width=\linewidth]{image/plot/CMRR_LNA1.png}
        \captionof{figure}{CMRR LNA1}
    \end{figure}  

    \begin{figure}[H]
        \centering
        \includegraphics[width=\linewidth]{image/plot/CMRR_LNA2.png}
        \captionof{figure}{CMRR LNA2}
    \end{figure}
\end{multicols}
\pagebreak
\subsection*{Power Suppy Rejection Ratio}

\begin{multicols}{2}
    \begin{figure}[H]
        \centering
        \includegraphics[width=\linewidth]{image/plot/PSRR_LNA1_VCC.png}
        \captionof{figure}{PSRR LNA1 VCC}
    \end{figure}  

    \begin{figure}[H]
        \centering
        \includegraphics[width=\linewidth]{image/plot/PSRR_LNA1_VCC.png}
        \captionof{figure}{PSRR LNA1 VCC}
    \end{figure}
\end{multicols}
\begin{multicols}{2}
\begin{center} 
        \includegraphics[width=\linewidth]{image/plot/PSRR_LNA2_VCC.png}
        \label{fig:PSRR_LNA2_VCC}
    \end{center}    
\begin{center} 
        \includegraphics[width=\linewidth]{image/plot/PSRR_LNA2_VEE.png}
        \label{fig:PSRR_LNA2_VEE}
    \end{center}
\end{multicols} 
\end{document}